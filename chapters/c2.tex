% !Mode:: "TeX:UTF-8"
\chapter{SDN IP网络下的并行业务量工程算法研究}
\section{引言}
SDN IP网络中的业务量工程问题可以被建模成一个多商品流问题,将网络业务作为问题的输入,寻找最优的路由来最小化代价函数,代价函数通常被设置为对网络拥塞程度的评价水平。比如,最常用的代价函数是最大链路利用率(MLU),简单地被定义为利用率最高的那条链路的链路利用率 \citing{MLU1, MLU2},另外一些文献 \citing{Convex1, Convex2}把所有链路的链路利用率的和作为代价函数,链路利用率代价函数的逻辑是:(1)低链路利用率意味着低的网络延迟。(2)维持低的链路利用率意味着预留更多的空间给其他将来到达的业务。但是大量基于实际拓扑的实验表明链路利用率代价函数,特别是MLU,在网络利用率没有达到拥塞程度的时候,不是对网络优化的较好评价函数 \citing{BeyondMLU}。在这个实验\citing{BeyondMLU}中,当链路利用率低于0.9的时候会造成不可忽略的网络性能中断。反之,在网络发生大量拥塞的情况下,MLU只去优化最大的链路利用率,而不能给出一个可行(满足容量约束)的解。所以作为替代,本文采用路由代价作为代价函数,我们假设已经知道短时间内到达的一批业务,控制器需要计算出满足链路容量约束的路径,并且最小化总的路由代价。为了使得加入网络的业务尽量多,我们设定被阻塞的业务代价的为一个较大值。

本章主要设计了两种业务量工程算法,第一种是基于备选路径模型的业务量工程算法,采用遗传算法来优化目标函数,并且设计了遗传算法的并行版本。第二种是基于拉格朗日松弛的优化算法,算法把链路容量约束松弛到目标函数,并把业务量工程问题分解成一批业务的路由计算问题,从而采用GPU进行并行计算。
\section{网络模型和问题建模}
\subsection{网络模型}

本文将SDN网络建模成有向图 $G(V, E)$,$V$表示所有的点集合,$E$是所有边的集合, $n = |V|$ 和 $m = |E|$ 分别表示点数和边数。对每一条边$(i,j)\in E$, $w_{ij}$ 表示此边 $(i,j)$上的权重(传输一单位的流量需要的代价),不失一般性,我们假设每条链路上的$w_{ij}$是整数,对每一条边 $(i,j)$, $c_{ij}$表示此边上的容量,假设$D$表示需要被路由的业务需求集合,业务$d \in D$ 是一个元组 $(s_d, t_d, bw_d)$,其中, $s_d$表示业务的源节点,$t_d$表示业务的目的节点,$bw_d$表示业务d需要的流量带宽。
\subsection{问题建模}
本小节,我们把业务量工程问题建模成一个混合整数规划模型(MILP),本文采用新的代价函数,如下式\ref{Obj1},\ref{obin}:
\begin{equation}\label{Obj1}
\sum\limits_{d \in D} f(\mathbf{d})
\end{equation}
\begin{equation}\label{obin}
f(\mathbf{d})=\begin{cases}
c(p_d) \cdot bw_{d} & \text{如果业务可以被加入}\\
W \cdot bw_{d}& \text{如果业务被阻塞}
\end{cases}
\end{equation}
其中$p_d$是计算出来的对应于业务$d$的路径,$c(p_d)$ ($c(p_d) = \sum_{(i,j)\in p_d} w_{ij}$) 表示的是此路径$p_d$ 的代价,$W$为一个较大的值,比如$W$远远大于业务所有可能的路径代价值。因为不知道带宽是否足够容纳所有业务,\ref{Obj1}有两个分支,为了使得表达式同一,方便表示,
我们首先构建辅助图$G_a(V_a, E_a)$ ,对每个点$v \in V$和$u \in V$,在$G_a(V_a,E_a)$中添加一条链路$(v,u)$,并且设置链路 $(u,v)$ 的容量和代价分别为$\infty$和$W$。然后,我们把$G_a(V_a,E_a)$和原图$G(V,E)$合并成一个新图$G_b(V_b,E_b)$,那么图$G_b(V_b,E_b)$就有足够的容量来容纳业务需求,如果某条业务被路由到$G_a(V_a, E_a)$的链路上,那么就表示这条业务被阻塞了,构建了辅助图$G_b(V_b, E_b)$后,业务量工程问题的代价函数可以表示为:
\begin{equation}\label{Obj2}
z^* = minimize~f(\mathbf{d})=
\sum\limits_{d \in D} c(p_d)\cdot bw_d= \sum\limits_{d \in D}\sum\limits_{e \in p_d} w_e \cdot bw_d
\end{equation}
在本文的业务量工程问题中,每个业务只能够路由到一条路径上,如第二章中讨论的那样,这更符合于SDN网络的实际情况,以下整数约束能够保证每个业务只走一条路径
\begin{equation}\label{FlowConv}
\begin{split}
\sum\limits_{(i,j) \in E_b} x_{ij}^d - \sum\limits_{(j,i) \in E_b} x_{ji}^d
=\begin{cases}
1 & \text{if $i = s_d$}\\
-1 & \text{if $i = t_d$} \\
0 &{ohterwise}
\end{cases}
\\~~~~~~~~\forall i\in V_a, \forall d\in D
\end{split}
\end{equation}
\vspace{-0.4cm}
其中 $x_{ij}^d$是一个01整数变量, $x_{ij}^d=1$表示业务$d$的路由经过了链路$(i,j)$,为了避免链路拥塞,路由需要满足以下的链路容量约束:
\begin{equation}\label{Capcon}
\sum\limits_{d \in D}x_{ij}^d \cdot bw_d \le c_{ij} ~~\forall (i,j)\in E_b
\end{equation}
在这个模型中,变量的数量随着业务量大小和网络规模大小呈倍数增长,所以这个MILP模型在大规模情况下很难求解。
\section{基于遗传算法的路由优化算法}
\subsection{备选路模型}
模型(\ref{Obj2},\ref{FlowConv},\ref{Capcon})的一种常见的简化模型被称为基于备选路径的模型\citing{multi-commodity},在备选路径模型中,为每一个业务$d \in D$,预先产生了$K$条不同的路径作为备选路径集合$P_d=\{p^1_d,p^2_d,p^3_d...p^K_d\}$,通过在这些备选路径中进行选择来优化目标函数\ref{Obj2},
为了表示业务被阻塞的情况,我们在业务的备选路径集合$P_d$中添加一条路径$p^0_d$,$p_d^0$是图$G_a(V_a,E_a)$上的路由,其代价为$W$,所经过的链路容量为$\infty$。于是,模型可以被描述如下:
\begin{equation}\label{Obj3}
z^* = minimize~f(\mathbf{d})=
\sum\limits_{k \in [0,K]}\sum\limits_{d \in D} c(p_d^k)\cdot x_k^d\cdot bw_d
\end{equation}
\begin{equation}\label{Cons1}
\sum\limits_{k \in [0,K]}x_k^d=1
\end{equation}
\begin{equation}\label{Cons2}
\sum\limits_{k \in [0,K]}\sum\limits_{d \in D} y_{ijk}^d\cdot x_k^d\cdot bw_d\le c_{ij} ~~\forall (i,j)\in E_b
\end{equation}
其中,$x_k^d$为一个01变量,当$x_k^d=1$时表示业务d选择其备选路径中的第k条路径,$y_{ijk}^d$是一个01变量,$y_{ijk}^d=1$表示业务$d$的第$k$条路径经过了链路$(i,j)$。约束\ref{Cons1}表示每个业务只能选择一条路径,约束\ref{Cons2}表示经过链路上的总流量大小不能超过链路上的容量大小。在这个模型中,由于我们把业务的路径限制在其备选路径集合中,所以他的变量个数比模型(\ref{Obj2},\ref{FlowConv},\ref{Capcon})要少很多,本节将通过遗传算法来求解这个简化的模型,设计出基于遗传算法的并行业务量工程算法GA-PTEA(Genetic Algorithm Based Parallel Traffic Engineering Algorithm)。

\subsection{遗传算法设计}
遗传算法是一种模拟自然进化过程来搜索问题最优解的启发式算法。遗传算法模仿达尔文进化论和自然选择过程来评价挑选最优解集合,遗传算法从一个代表问题的可行解的种群出发,一个种群中不同个体代表了不同的解,每个个体实际上是一个染色体,染色体携带表达当前解的信息编码,初代种群产生后,对每个染色体个体进行评价,按照适者生存,优胜劣汰的原则,使得优秀的个体更有可能把自己的遗传信息传递给下一代,从而得到更优秀的后代。在遗传算法的迭代过程中,一部分基因会发生变异,好的变异能够提高解的质量,增加算法的搜索空间,避免算法收敛于局部最优解。遗传算法的算法流程图如图\ref{IterNum} 所示,遗传算法先从初始种群开始,循环的地进行交叉,变异,评价选优这三种操作,最终得到一个最优秀的个体作为问题的解。
\begin{figure}
\setlength{\abovecaptionskip}{-0.5cm}
\begin{center}
{\includegraphics[width=0.8 \textwidth]{figures/GAprocess.pdf}}
\end{center}
\caption{{\footnotesize{遗传算法流程图}}}
\label{IterNum}
\end{figure}
\subsubsection{染色体结构}

假设业务数量为$|D|$,初始染色体集合大小为$POP$,集合$C$表示染色体种群集合,$C_j$表示第$j$($j \in [1,POP]$)号染色体,$C^i_j=k,(k \in [1,K])$表示在第$j$号染色体中,业务$i$选择了第$k$条备选路径,$C^i_j=0$表示业务$i$被阻塞(在辅助图$G_a(V_a,E_a)$上路由),不占用链路资源,$|p^i_d|$表示业务$d$的第$i$ 条备选路的路由代价,$rp^i_d$ 表示路径$p^i_d$ 上的最小可用容量,$rp^i_d=min(r_e|e \in p^i_d)$,其中$e \in p^i_d$ 表示路径$p^i_d$上的边,$r_e$ 表示此边$e$ 上的剩余容量。
\subsubsection{初始可行解的生成}
可行解表示满足容量约束的解,为了使得遗传算法有效,初始解的质量很重要,产生的初始解要尽量好,这表示要有更多的业务要加入网络中,而且保证业务的路由代价较小,本文采用一种简单的贪心算法来产生出初始可行解,算法如下\ref{alg:Framwork}所示:
\begin{algorithm}[htb]
\begin{algorithmic}[1]
\Require{$G(v,E)$:网络拓扑;$P$:备选路径集合;$C$:未初始化的染色体集合;}
\Ensure{$C$:可行染色体集合;}
\For{$c_j \in C$}
\For{$c^d_j \in c_j$ }
\State {$c^d_j\leftarrow k^d_j$,其中$k^d_j$为$1$到$K$之间的随机值,随机选择一条备选路}
\EndFor
\State {对染色体中的的每个业务需求按照值$\frac{bw_d}{\sqrt{|p^{k^d_j}_d|}}$}进行降序排序。
\For{$c^d_j \in c_j$ }
\If{$rp^{k^d_j}_d \ge bw_d$}
\State 加入路径$p^{k^d_j}_d$到网络,更新网络链路容量。
\Else
\State {$c^d_j \leftarrow 0$}
\EndIf
\EndFor
\EndFor
\end{algorithmic}
\caption{初始可行解生成算法}
\label{alg:Framwork}
\end{algorithm}

对每一个染色体,算法\ref{alg:Framwork}随机地为每个业务选择备选路径标号,但是这样选择出来的路径集合有可能会超过网络链路的容量限制,从而使得解变得不可行,要得到可行解,必须从解中剔除一部分业务。使得他们阻塞,本文使用一种启发式策略来确定需要剔除的业务。一方面,要使得目标函数变小,那些流量需求较大的业务应该优先被加入到网络中。但是,另一方面,如果大流量的业务的路由代价很大,经过了一条很长的路径,就会大量地浪费网络中的链路容量资源。所以算法对当前染色体$j$中的业务和其路径按照$\frac{bw_d}{\sqrt{|p^{k^d_j}_d|}}$ 的值进行排序,其中${bw_d}$ 代表当前业务$d$所需要的流量大小,$|p^{k^d_j}_d|$代表集合$P_d$ 中的第${k^d_j}$条路径的代价大小,观察目标函数,目标函数是最小化路由代价,而$\frac{bw_d}{\sqrt{|p^{k^d_j}_d|}}$较大意味着较大的流量经过较小代价的链路进行路由,这种路由是很理想的,尽量节省网络的链路使用资源的同时,又减小了总体目标函数,所以这个比例值是对业务路由优劣程度的较好评价。算法按照比例值排序的结果依次尝试将路径$p^{k^d_j}_d$加入到网络中,如果$rp^{k^d_j}_d \ge bw_d$,表示路由经过的链路有足够的容量来容纳这一业务,所以加入业务到网络中,并且更新网络的链路容量大小,反之,如果$rp^{k^d_j}_d<bw_d$,这个业务选择这一条路径会超过网络链路的容量限制,于是这条业务被阻塞,染色体中的相应基因位置被设置为0。我们重复上面的过程$POP$次,就得到一个初始可行染色体集合$C$。
\subsubsection{评价}

评价过程对每一个染色体计算其相应的目标代价函数值,并且对染色体按照代价函数值进行降序排序,由于遗传算法中的交叉和变异步骤可能会产生不可行的染色体解(链路容量超限),我们把这样的染色体评价为一个很大的代价(INF),从而在选优时被排除掉。我们把排序好的染色体分为三组:

精英染色体集合:目标值排在最前面的$\alpha$个染色体构成精英染色体集合$A$,精英染色体将直接保留到下一轮迭代,而且精英染色体可以产生后代。

次优染色体集合:精英染色体后的$\beta$个染色体构成次优染色体集合$B$,次优染色体集合中的染色体不会直接保留到下一轮,但是他们可以和精英染色体进行交叉产生后代。

劣等染色体集合:最后剩余的$\gamma$个染色体构成劣等染色体集合$G$,由于其代价函数值一般较大,其选路策略不可取,算法直接扔掉这一部分劣等染色体,不允许劣等染色体产生后代。
\subsubsection{交叉}
交叉过程从精英染色体集合$A$中随机地选取一个染色体$c_i \in A$作为父亲,从精英染色体集合和较优染色体集合的并集$A \cup B$ 中随机选取一个染色体$c_j \in A \cup B$ 作为母亲,将$c_i$ 和$c_j$进行均匀交叉得到新的染色体$s$。均匀交叉的过程是,对$s$的每一个基因点位以$\%50$的几率选择继承父亲或者母亲的相应点位的路径选择。重复以上过程$\beta$ 次,从而产生$\beta$个新的子染色体。
\subsubsection{变异}
变异过程采用随机变异,随机在集合$A \cup B$中选取$\gamma$条染色体,对某一选定的染色体$c_j \in A \cup B$,随机选取$m$个业务基因点位,进行变异,将当前已经选择的路径编号随机改变为备选路集合中的另外一个值,变异过程能够提高算法的的搜索空间,避免算法陷入局部最优解。
\subsubsection{终止条件}
假设第前$k$次迭代找到的最优解为$B^*$,第k+1次迭代找到最优解为$b_{k+1}^*$,如果$b_{k+1}^*<B^*$,那么就更新$B^*=b_{k+1}^*$,如果连续迭代$L$次,$B^*$都不被更新,则判定算法收敛,算法停止。
\subsection{基于GPU的并行遗传算法设计}
\subsubsection{并行评价算法设计}
遗传算法中最消耗时间的部分是染色体评价部分,由于需要评价大量的染色体,评价每个染色体都需要大量的计算开销。但是幸运的是遗传算法具有天然的并行性,每个不同的染色体评价可以并行执行,更进一步,每个染色体中的不同基因的计算也可以并行执行,这样并行粒度是达到$|POP|\times|D|$。在具体介绍并行算法之前,我们先引入一些符号,如下:

染色体(chromosomes)相关符号:$C$表示染色体种群集合,$c_j$表示第$j$个染色体,$c^d_i$表示第$i$个染色体的第$d$个基因位置。

备选路径(paths)相关符号:$P$为所有业务的备选路径组成的集合,$P_i$表示业务$i$的所有备选路径集合,$p_i^k \in P_i$表示集合$P_i$中的第$k$条路径。

业务带宽(bandwidth,bw),$bw_i$表示第$i$个业务需要的带宽大小。

链路流量(flow,f),$f_e$表示链路$e$上占用的流量大小。

链路单位代价(weight,w),$w_e$表示链路$e$上的代价。

链路容量(capacity,ca),$ca_e$表示链路$e$上的容量大小。

共享内存中间数组(shared,sh),$sh_e$表示链路$e$上的总代价。

目标函数值数组(objectives,ob),$ob_j$表示第$j$个染色体对应的目标函数值。

\begin{algorithm}[t]
\begin{algorithmic}[1]
\Function{evaluate}{$C$,$P$,$bw$,$w$,$ca$,$ob$}
\State {在block上分配大小为|E|的共享内存空间组成数组f,初始化数组f中的所有值为零。}
\State {$j \leftarrow$ block ID}
\State {$d \leftarrow$ thread ID}
\State {$k \leftarrow c_j^d$}
\For {$e \in p_d^k$}
\State {$atomicAdd(f_e,bw_d)$}
\EndFor
\State{调用block线程同步函数\_\_syncthreads()}
\State {在block上分配大小为|E|的共享内存空间组成数组sh,初始化数组sh中的所有值为零。}
\State{$e \leftarrow thread ID$}
\If {$f_e \le ca_e$}
\State{$sh_e \leftarrow f_e\cdot w_e$}
\Else
\State{$sh_e \leftarrow INF$}
\EndIf
\State{调用block线程同步函数\_\_syncthreads()}
\State{$s \leftarrow |E|$}
\While{s>1}
\If{e<s/2}
\State{$sh_e \leftarrow sh_e+sh_{(e+(s+1)/2)}$}
\EndIf
\State{调用block线程同步函数\_\_syncthreads()}
\State{$s \leftarrow (s+1)/2$}
\EndWhile
\State{$ob_j=sh_0$}
\EndFunction
\end{algorithmic}
\caption{kernel 函数evaluate}
\label{evaluate}
\end{algorithm}

由于每个染色体的计算过程是独立的,算法为每一个染色体开辟一个block,一个block内部的第$d$号线程负责当前染色体上第$d$号业务的计算。算法 \ref{evaluate}展示了负责评价的kernel函数evaluate,算法一共可以分为两个部分:

第一,流量统计部分:算法需要对染色体进行可行性判断,也就是判断是否有链路的流量大小超过其容量大小,为此,算法需要统计每一条边上的流量大小,我们在共享内存中开辟数组$f$用以统计流量,$f$数组被初始化为0。一个block中的每一个线程首先通过寻址备选路径集合找到业务选择的路径,
业务每经过一条链路都需要占用该链路上的容量,所以如果路径经过了链路$e$,$f_e$就需要被加上此业务的流量大小。block中的每一个线程都会遍历业务所选择的路径,对$f$数组进行加法操作,但是多个线程可能同时对$f$数组中的同一个地址进行加法操作,这不是同步安全的,所以算法使用atomicAdd操作进行原子加法,避免同步问题。
当线程完成对$f$数组的统计之后,必须进行同步(syncthreads()),同步操作保证block内部的所有线程执行到同一步骤,也就是先统计完成的线程必修等待其他统计线程也执行完成后才能继续执行,因为只有所有线程都完成对$f$数组的加法操作,$f$数组的值才完整,才能够进行下一步操作。

第二,代价函数计算部分:代价函数为每条链路的路由代价之和,算法需要先计算每一条链路上的代价值,我们在共享内存中开辟数组$sh$来对每一条边的代价进行统计,每一个线程负责对一条链路的路由代价进行计算。比如,线程$e$负责第$e$条边的代价计算,线程$e$首先判断链路的流量是否溢出($ca_e<f_e$),如果溢出则说明此染色体表示的解不可行,设置$sh_e$为$\infty$,这使得这个染色体在排序时会被归类到劣等集合,从而在下一次迭代之前被剔除掉。反之,如果$f_e \le ca_e$,则说明链路容量足够,链路的代价为$f_e \cdot w_e$,设置$sh_e$为$f_e \cdot w_e$。和$f$数组的统计一样,当线程统计完成$sh_e$之后,也需要进行同步操作,等待其他线程完成统计。$sh_e$数组计算完成之后,还需要把$sh_e$中的值进行加和才能得到效用函数的值,为了充分利用GPU多线程,算法最后使用并行规约对数组$sh$(20-26行)进行求和,for循环中每次将后一半的$sh$数组加到前一半,规约过程中必须进行同步(syncthreads()),以保证加法过程计算完整,最终求和值规约到$sh_0$,将$sh_0$中的值写入到$ob$数组中。
\begin{figure}
\vspace{-0.8cm}
\setlength{\abovecaptionskip}{-0.5cm}
\begin{center}
{\includegraphics[width=1 \textwidth]{figures/GPUfitness.pdf}}
\end{center}
\caption{{\footnotesize{GPU并行评价示例}}}
\label{fitness}
\end{figure}

图\ref{fitness}为GPU上并行评价算法的一个具体例子,图中被标位蓝色的数组表示业务选择的路径,蓝色的箭头表示流量统计时的加法操作,红色箭头表示链路流量超过链路容量,使得链路代价被设置为$\infty$。
\subsubsection{并行排序,变异与交叉}
在评价部分结束后,需要对所有的染色体按照效用函数的大小降序排序,本设计采用CUDA库Thrust\citing{Thrust}中提供的排序函数对GPU上的染色体进行排序,Thrust是基于标准模板库(STL)的CUDA C++模板库,Thrust为程序员提供常见的CUDA算法库,能够减少程序员工作量并且提高应用程序效率,Thrust库中的排序函数已经针对GPU架构做了很好得优化,所以本文不在对排序部分进行优化,排序函数细节可参考NVDIA官方文档\citing{Thrust}。

交叉过程分为父母选取和交叉计算两个kernel,父母选取过程并行地选取$\beta$对父母,每个线程负责选取一对父母,并且将选取的父母下标记录到$father$和$mother$数组中,父母选取GPU计算示意图如图\ref{pp}所示。
\begin{figure}
\vspace{-1cm}
\setlength{\abovecaptionskip}{-0.5cm}
\begin{center}
{\includegraphics[width=0.8\textwidth]{figures/GPUchoose.pdf}}
\end{center}
\caption{{\footnotesize{GPU并行父母选取过程}}}
\label{pp}
\end{figure}
每个线程进行两次随机,$mother$的标号只能在前$\alpha$个精英染色体中选取,$father$的标号只能在前$\alpha+\beta$中选取,这样既能够给予精英染色体更大的繁殖几率,也可以保证解的搜索空间变化足够大,避免陷入局部最优值。

父母选取过程结束后,在GPU端记录了所选取的$mother$和$father$数组,$mother$和$father$数组将用于交叉部分的计算,交叉部分的并行粒度更高,其中每个基因点的计算都是并行执行的,如图 \ref{cross} 所示,
\begin{figure}
\vspace{-1cm}
\setlength{\abovecaptionskip}{-0.5cm}
\begin{center}
{\includegraphics[width=0.8\textwidth]{figures/GPUcross.pdf}}
\end{center}
\caption{{\footnotesize{GPU并行均匀交叉过程}}}
\label{cross}
\end{figure}
每个block负责一个新染色体的生成,通过之前的分析,一共需要生产$\beta$个新的染色体,所以GPU端一共需要分配$\beta$ 个block,其中每一个block中有|D|个线程来同时负责随机从$mother$和$father$的相应点位选择一个来作为新染色体相应点位的值,我们采用均匀交叉策略,选择父亲母亲遗传基因的概率都是50\%,这样block内部的并行度达到$|D|$,总的并行粒度达到$\beta \cdot |D|$。

最后,对于变异部分,我们一共开辟$\gamma$个线程,每个线程随机在$A\cup B$集合中选取一个染色体,并随机选取$m$个点位进行变异,并行粒度为$\gamma$。
\section{基于拉格朗日的优化算法设计}

GA-PTEA算法虽然简单,但是其具有以下缺点:第一,需要事先计算大量备选路径,不能很好地适应网络的动态变化,一旦网络链路发生变化,又要重新计算备选路径。第二,遗传算法从开始到收敛需要大量的迭代次数,虽然经过GPU加速,但是由于收敛缓慢,仍然需要大量的计算时间。第三,遗传算法容易陷入局部最优解。

本节采用基于拉格朗日松弛的模型来解决业务量工程问题,并根据这个模型设计出并行算法LR-PTEA(Lagrange Relaxing Based Parallel Traffic Engineering Algorithm),LR-PTEA相对与GA-PTEA有以下优点:第一,LR-PTEA能够快速收敛。第二,LR-PTEA求得的解大大优于GA-PTEA。第三,LR-PTEA不需要事先产生备选路径,业务路由是实时计算出来的。
\subsection{基于拉格朗日松弛的模型}
在模型(\ref{Obj2},\ref{FlowConv},\ref{Capcon})中,网络容量约束(式子 \ref{Capcon}),把所有的路由变量联系到一起,因为这些变量的取值必修保证每一条链路上占用的流量小于链路的容量大小,正是由于存在链路容量约束,每个业务的路由选取才变得不相互独立,但是要利用GPU 的并行特性,需要寻找独立计算的可能性,因此,本文采用拉格朗日松弛方法,把一个业务量工程问题分解成一批业务的路由计算问题,而这些业务的路由计算问题是相互独立的,很适合并行计算。

将模型(\ref{Obj2},\ref{FlowConv},\ref{Capcon})中的网络容量约束松弛进目标函数得到如下拉格朗日子问题:
\begin{equation}\label{LagProb}
\begin{split}
L(\mathbf{\lambda})= min\sum\limits_{d \in D}\sum\limits_{(i,j) \in E_b} w_{ij}\cdot x_{ij}^d\cdot bw_d+ \\ ~~~~~\sum\limits_{(i,j) \in E_b}\lambda_{ij}(\sum\limits_{d \in D} x_{ij}^d \cdot bw_d-c_{ij})
\end{split}
\end{equation}
其中 $\lambda_{ij}$ 表示链路 $(i,j)$的拉格朗日乘子。

表达式(\ref{LagProb})还可以表示为:
\begin{equation}\label{Lagprob1}
L(\mathbf{\lambda})= min\sum\limits_{d \in D}\sum\limits_{(i,j) \in E_b} (w_{ij}+\lambda_{ij})x_{ij}^d \cdot bw_d -\sum\limits_{(i,j) \in E_b}\lambda_{ij}c_{ij}
\end{equation}
受限于:
\begin{equation}\label{FlowConv2}
\begin{split}
\sum\limits_{(i,j) \in E_b} x_{ij}^d - \sum\limits_{(j,i) \in E_b} x_{ji}^d
=\begin{cases}
1 & \text{if $i = s_d$}\\
-1 & \text{if $i = t_d$} \\
0 &{otherwise}
\end{cases}
\\~~~~~~~~\forall i\in V_b, \forall d\in D
\end{split}
\end{equation}
拉格朗日子问题的目标函数中的$\sum_{(i,j) \in E_a}\lambda_{ij}c_{ij}$这一项,不随着拉格朗日乘子的变化而变化,本文将其作为常数项而丢掉不讨论,丢掉$\sum_{(i,j) \in E_a}\lambda_{ij}c_{ij}$ 这一项后,拉格朗日子问题的目标函数中只含有代价部分 $w_{ij}+\lambda_{ij}$和$x_{ij}^d\cdot bw_d$的乘积。注意到,$\sum_{(i,j) \in E_a} (w_{ij}+\lambda_{ij})x_{ij}^d\cdot bw_d$表示业务$d$的路由代价,因此,拉格朗日子问题的目标函数是最小化所有业务的路由代价之和,观察这个子问题的约束,我们发现,每一个约束都只含有一个和业务需求相关的变量,所以,这个拉格朗日子问题可以被分解成一系列独立的最短路径问题(每个业务需求对应于一个最短路问题),只是这些最短路径问题的链路代价发生了改变,链路代价变得和拉格朗日乘子 $\mathbf{\lambda}$ 相关,也就是说给定一个拉格朗日乘子$\mathbf{\lambda}$,我们可以将拉格朗日子问题看成一批单业务的最短路径问题,我们可以通过并行地计算一系列的最短路径来解决这个拉格朗日子问题。

因为把容量约束松弛进代价函数中后,不会增加目标函数的值, $L(\mathbf{\lambda})$成为原问题最优目标函数值的下界,$z^* \ge L(\mathbf{\lambda})$,为了得到最紧的的下界值,我们要解决以下这个优化问题:
\begin{equation}\label{dual}
L^*(\mathbf{\lambda^*}) = maximize_{\mathbf{\lambda}}L(\mathbf{\lambda})
\end{equation}
~~受限于: (\ref{FlowConv2})
\vskip 0.2cm

以上的这个优化问题也被称为原来业务量工程问题(式子\ref{Obj2},\ref{FlowConv},\ref{Capcon})的对偶问题 \citing{NetworkFlow},其中$\mathbf{\lambda^*}$ 表示最优拉格朗日乘子,为了得到最优乘子$\mathbf{\lambda^*}$,可以使用次梯度优化算法来解决,次梯度优化计算时,第一次先初始化乘子 $\mathbf{\lambda}^0$, 然后通过以下过程进行迭代求解:
\begin{equation}\label{iter}
\mathbf{\lambda}_{ij}^{(k+1)} =\mathbf{\lambda_{ij}}^{(k)}+\theta_{k} g^{(k)}= \mathbf{\lambda_{ij}}^{(k)} + \theta_k[(\sum\limits_{d \in D}x_{ij}^d\cdot bw_d- c_{ij})]^+
\end{equation}
其中, $\mathbf{\lambda}_{ij}^{(k)}$表示第$k$次迭代的对应于边$(i,j)$的拉格朗日乘子, $g^{(k)}$是$L(\mathbf{\lambda})$ 对$\mathbf{\lambda}^{k}$的任意一种次梯度,$\theta_k$ 表示第$k$次的迭代的步长,标记$[\alpha]^+$ 表示$\alpha$中符号为正的部分,也就说$[\alpha]^+=max(\alpha, 0)$,从表达式Eq. (\ref{iter})可以看出来如果链路$(i,j)$ 上的流量总和超过链路$(i,j)$上的容量,链路$(i,j)$ 上的$\lambda_{ij}^k$拉格朗日乘子会增加,也就是表示一些业务流量需要从链路$(i,j)$上移除,另外,为了避免产生负权重的链路代价,当链路容量大于其上的流量时,我们并不去减小此链路$(i,j)$上的$\lambda_{ij}^k$。

根据以上讨论,我们给出基于拉格朗日乘子法的并行业务量工程算法的框架,如图\ref{lpl}所示,LR-PTEA主要包括以下步骤:

步骤一,为$G(V, E)$初始化链路权重。

步骤二,计算所有业务的最短路径,其中路径计算任务被分配到GPU上进行并行计算。

步骤三,为了从当前计算出来的路径中得到原问题的优化目标函数值,对步骤二中计算出来的路径进行调整。

步骤四,更新链路权重,更新完毕后,如果停止条件不满足,则回到步骤二,进入下一轮迭代。
\begin{figure}
\vspace{-1cm}
\setlength{\abovecaptionskip}{-0.5cm}
\begin{center}
{\includegraphics[width=1\textwidth]{figures/lagrange.pdf}}
\end{center}
\caption{{\footnotesize{LR-PTEA算法流程图}}}
\label{lpl}
\end{figure}
\subsection{基于GPU的并行路由计算}
在每次迭代中,LR-PTEA为每个业务 $d \in D$在图 $G(V, E)$中计算最短路径,显然,丢掉链路容量约束后,不同业务的最短路径计算可以独立地在GPU上并行执行,但是,最短路算法的逻辑对于GPU来说太过复杂,GPU最初是被设计来做大规模的数值计算问题,其只实用于逻辑比较简单,但是数值计算量较大的任务,所以在GPU上直接开辟一个线程来计算一个业务的路径,不仅仅在计算上是低效的,而且这样的并行粒度也不能充分利用GPU 的大规模并行能力。为了提高并行粒度,LR-PTEA对最短路径算法也进行并行化设计。

文章 \citing{SSP1}提出一种Dijkstra最短路径算法在GPU上的并行实现,但是从算法结构上分析,Dijkstra最短路径算法并不适应于并行算法的设计,所以Dijkstra最短路径算法在GPU上的实现不能得到很好的加速效果。为了得到更好的加速效果,LR-PTEA选择Bellman-Ford \citing{NetworkFlow}最短路算法来进行并行实现。Bellman-Ford最短路算法逐步地减小距离标记 $Dist[v]$($v\in V$),直到其收敛到真实的最短距离。串行的Bellman-Ford算法流程如算法\ref{Bellman}所示,其中 $Dist[v]$ 表示距离起点$s$到$v$的最短路径距离,$Pre[v]$ 表示点$s$到点$v$ 的最短路径上$v$ 的前驱节点,在初始化好了所有节点的距离数组和前驱节点数组之后,Bellman-Ford算法最多迭代 $|V|$ 次,每一次迭代,算法都会松弛图$G(V, E)$中的所有的边(第10-11行)。Bellman-Ford 算法的算法复杂度为$(|V|\cdot |E|)$,他的复杂度一般是高于Dijkstra最短路径算法的复杂度,但是,因为Bellman-Ford算法每次松弛边的操作都是独立无关的,Bellman-For算法很容易在GPU上实现并行化。

图 \ref{ParFramework} 中显示了LR-PTEA的最短路径计算的并行实现框架。首先,根据业务的源节点将业务分配成不同的组,使得每一组内业务的源节点相同。我们假设第$i$个组的源节点为$s_i$。 然后,每一组的最短路径计算使用$m$个GPU 线程的并行地进行边的松弛操作,如图\ref{ParFramework}所示,线程$T_{i,j}$负责为对应于源点$s_i$的链路$e_{j}$ 进行松弛操作。因此总的并行执行的线程数是$m \times k$,其中$m$和$k$分别表示链路数目和组的数目。

在本文的最短路算法的并行实现中,每个block内部的线程都松弛同一条链路,比如,在图 \ref{GB} 中,集合$\{T_{1j}, T_{2j}, \cdots, T_{ij}, \cdots, T_{kj}\}$都在$block_j$上执行,其中 $T_{ij}$为对应源节点为$s_i$ 的链路$e_i$执行松弛操作,其中,我们设链路$e_i$的头节点和尾节点分别为$h_i$ 和$t_i$。可以看到,当存在链路更新时,标记$Mark$会被设置成1,这是为了优化算法的迭代次数,当某次迭代结束$Mark=0$ 则表示这次迭代没有边进行了更新操作,说明Bellman 算法已经提前结束,我们的实验证明这一个优化可以大大地减小Bellman算法的迭代运行次数。
\begin{algorithm}[htb]
\begin{algorithmic}[1]
\Require
网络拓扑:$G(V, E)$;
源点:$s$;
\Ensure
从$s$开始到其他点的路径集合$P$;
\For {each node $v \in V$}
\State {$Dist[v] \leftarrow \infty$}
\State {$Pre[v] \leftarrow $ NIL}
\EndFor
\State {$Dist[s] \leftarrow 0$}
\State {$Mark \leftarrow 1$}
\While {$Mark > 0$}
\State {$Mark \leftarrow 0$}
\For{each link $(u,v) \in E$}
\If{$Dist[v]>Dist[u]+w_{uv}$}
\State {$Dist[v] \leftarrow Dist[u]+w_{uv}$}
\State {$Pre[v] \leftarrow u$}
\State {$Mark = 1$}
\EndIf
\EndFor
\EndWhile
\State {根据前驱数组$Pre$,重新构建最短路集合,输出路径到集合$P$}
\end{algorithmic}
\caption{{Bellman-Ford最短路算法}}
\label{Bellman}
\end{algorithm}


\begin{figure*}
\vspace{-0.5cm}
\setlength{\abovecaptionskip}{-0.5cm}
\begin{center}
{\includegraphics[width=1 \textwidth]{figures/paframework.pdf}}
\end{center}
\caption{{\footnotesize{并行业务计算框架}}}
\label{ParFramework}
\end{figure*}




\begin{figure*}
\vspace{-0.4cm}
\setlength{\abovecaptionskip}{-0.5cm}
\begin{center}
{\includegraphics[width=1 \textwidth]{figures/GPUimpl.pdf}}
\end{center}
\caption{{\footnotesize{GPU上Bellman算法的实现}}}
\label{GB}
\end{figure*}

\begin{figure*}
\setlength{\abovecaptionskip}{-0.5cm}
\begin{center}
{\includegraphics[width=0.8 \textwidth]{figures/SynPro.pdf}}
\end{center}
\caption{{\footnotesize{同步问题的例子}}}
\label{SynPro}
\end{figure*}
多业务的并行Bellman算法的CUDA实现如算法\ref{kd}所示。需要注意的是由于线程在GPU上是独立执行的,在更新节点的距离标记和前驱标记的时候会出现同步问题,假设线程$T_1$为链路$(x,v)$执行松弛操作,而线程$T_2$为链路$(y,v)$ 执行松弛操作,假设两个线程更新点$v$ 的距离标记和更新前驱标记的顺序如图\ref{SynPro} 所示,如果$Dist[y] + w(y, v) < Dist[x] +w(x, v)$,那么$Dist[v]$ 被更新为$Dist[y] + w(y, v)$,但是由于更新的顺序发生交叉,节点$v$的前驱节点被更新成了$x$, 而不是真正正确的前驱节点$y$。为了避免这个同步问题,我们使用两个kernel,一个kernel\_distance\_update(算法\ref{KernelDist})用来更新距离标号,一个kernel\_distance\_update(算法\ref{KernelPre})用来更新前驱节点。
\begin{algorithm}[t]
\begin{algorithmic}[1]
\Require
业务需求集合$D$;
链路集合$E$;
\Ensure {业务需求的最短路径集合$P$}
\State {将业务的源节点加入到集合$S$中}
\State {$Mark \leftarrow$ 1}
\While{$Mark > 0$}
\State {$Mark \leftarrow$ 0}
\State {发射 kernel\_distance\_update($S$, $E$, $Dist$)}
\EndWhile
\State {发射 kernel\_predecessor\_update($S$, $E$, $Dist$, $Pre$)}
\State {根据前驱数组$Pre$重建业务的最短路径,并把路径加入到集合$P$中}
\end{algorithmic}
\caption{{并行最短路计算过程}}
\label{kd}
\end{algorithm}

\begin{algorithm}[t]
\begin{algorithmic}[1]
\Function{kernel\_distance\_update}{$S$, $E$, $Dist$}
\State {$bid \leftarrow$ block ID}
\State {$tid \leftarrow$ thread ID}
\State {将 $(bid, tid)$ 映射到 id $sid$}
\State {$s \leftarrow S[sid]$}
\State {$e \leftarrow E[bid]$}
\If{$Dist[s][e.tail] + e.weight < Dist[s][e.head]$}
\State {$Mark \leftarrow 1$}
\State {$Dist[s][e.head] \leftarrow Dist[s][e.tail] + e.weight$}
\EndIf
\EndFunction
\end{algorithmic}
\caption{kernel 函数kernel\_distance\_update}
\label{KernelDist}
\end{algorithm}

\begin{algorithm}[t]
\begin{algorithmic}[1]
\Function {kernel\_predecessor\_update}{$S$, $E$, $Dist$,$Pre$}
\State {$bid \leftarrow$ block ID}
\State {$tid \leftarrow$ thread ID}
\State {将 $(bid, tid)$ 映射到 id $sid$}
\State {$s \leftarrow S[sid]$}
\State {$e \leftarrow E[bid]$}
\If{$Dist[s][e.tail] + e.weight = Dist[s][e.head]$}
\State {$Pre[s][e.head]= e.tail$}
\EndIf
\EndFunction
\end{algorithmic}
\caption{kernel函数 kernel\_predecessor\_update}
\label{KernelPre}
\end{algorithm}
\subsection{链路权重更新}
\subsubsection{权重更新步长}
在LR-PTEA算法中,在第$k+1$次迭代,链路$(i,j)$的权重被更新为$w_{ij}^{k} + \lambda_{ij}^{k+1}$, 其中$\lambda_{ij}^{k+1}$ 被更新为:
\begin{equation}\label{Iter}
\lambda_{ij}^{k+1} = \lambda_{ij}^k + \theta_k[(\sum\limits_{d \in D}x_{ij}^dbw_d - c_{ij})]^+.
\end{equation}
为了保证收敛性,第$k$次迭代的更新步长$\theta_k$可以被设置为 $\frac{1}{k}$ \citing{NetworkFlow}。 然而,通过仿真发现,当$\theta_k$被设置为 $\frac{1}{k}$时,其收敛缓慢。让我们考虑图 \ref{u1}的例子,其中链路$(i,j)$ 上的标记分别表示链路权重和链路上剩余的容量大小。假设现在有两个业务需求$d_1$ 和$d_2$,他们的源都是$A$,目的节点都是$D$,且每一个业务的流量大小都是4个单位。为了展示这个迭代过程,我们把算法前5次的迭代结果表示在表 \ref{Iterprocess1}中,从表中可以看到,业务计算出的最短路径一直在$A-B-D$和$A-C-D$ 之间徘徊。算法必修等到$A-B-D$ 和$A-C-D$ 的两条路的权重相等时才能停止,只有这样这两个业务才可能分离,其中一个选择$A-B-D$,而另一个选择$A-C-D$。 但是,正如表中所示这需要大量的迭代才能达到。
\begin{figure}
\setlength{\abovecaptionskip}{-0.5cm}
\begin{center}
{\includegraphics[width=0.4 \textwidth]{figures/IterNum.pdf}}
\end{center}
\caption{{\footnotesize{链路更新例子(1)}}}
\label{u1}
\end{figure}
\begin{table}[t]
\newcommand{\tabincell}[2]{\begin{tabular}{@{}#1@{}}#2\end{tabular}}
\setlength{\abovecaptionskip}{0.2cm}
\centering
\scriptsize{
\renewcommand{\tabcolsep}{0.09cm}
\renewcommand{\arraystretch}{1.2}
\caption{拉格朗日更新过程(1)}
\begin{tabular}{| c | c | c | c|}
\hline
Iteration number & \tabincell{l}{Calculated paths for \\the two demands} & Path weights & $\theta_k$
\\ \hline
0 & \tabincell{l}{A-B-D \\ A-B-D} & \tabincell{l}{2(1+1) \\ 2(1+1)} & 1 \\ \hline
1 & \tabincell{l}{A-C-D \\ A-C-D } & \tabincell{l}{4(2+2) \\ 4(2+2)} & 1 \\ \hline
2 & \tabincell{l}{A-B-D \\ A-B-D} & \tabincell{l}{10(1+4+1+4) \\ 10(1+4+1+4)} & 0.5 \\ \hline
3 & \tabincell{l}{A-C-D\\ A-C-D} & \tabincell{l}{12(2+4+2+4) \\ 12(2+4+2+4)} & 0.33 \\ \hline
4 & \tabincell{l}{A-B-D \\ A-B-D} & \tabincell{l}{14(1+6+1+6) \\ 14(1+6+1+6)} & 0.25 \\ \hline
5 & \tabincell{l}{A-C-D \\ A-C-D} & \tabincell{l}{14.66(2+5.33+2+5.33) \\ 14.66(2+5.33+2+5.33)} & 0.2 \\ \hline
\end{tabular}
\vskip 0.2 cm
\label{Iterprocess1}
}
\end{table}

另外一种常用的步长选择是:
\begin{equation}\label{StepSize}
\theta_k = \frac{\rho[UB-{L(\mathbf{\lambda}^k)]}}{||\mathbf{Ax^k}-\mathbf{b}||^2}
\end{equation}
其中$UB$是最优化目标函数的上界,$\rho$是一个取值范围为0到2的常数,$\mathbf{A}$ 和 $\mathbf{b}$分别是链路相关矩阵和链路上的剩余容量向量。但是从实验中发现设置这种步长迭代效果也不令人满意。

本设计采用一种简单但是有效的链路权重更新步长,假设$\theta_{k}^{ij}$ 为第$k$次迭代时链路$(i,j)$ 上的需更新的步长,那么$\theta_{k}^{ij}$ 为:
\begin{equation}\label{StepSizeUsed}
\theta_{k}^{ij} = \frac{1}{|c_{ij}-\sum\limits_{d \in D} x_{ij}^d bw_d|}
\end{equation}

从式子 \ref{StepSizeUsed},我们可以看到,如果一条链路上承载的流量大小超过了这条链路上的容量大小,那么这条链路上的权重在下一次迭代之前就会增加1,对于其他的流量满足约束的链路,其权重不会改变,如果使用式子\ref{StepSizeUsed}的步长更新方法,LR-PTEA仅仅只需要一次迭代就能够得到例子(图\ref{u1})中的最优的权重。我们的实验表明,这种粗粒度的更新操作大大的减小了算法的收敛迭代次数,从而大大缩短算法运行时间。
\subsubsection{随机更新策略}

拉格朗日松弛法将原问题分解成了一个个独立的最短路径问题,这样使得算法可以并行化进行设计。但是由于每个子问题独立分离,使得每个问题在求最短路径时都是贪心的,这样可能会使得大量业务抢占同一批链路,造成拥塞,一旦拥塞,链路的权重就会增加,又会使得大量的业务集体放弃这一批链路,去抢占其他链路,使得其他链路也发生拥塞,形成一种恶性循环,最终会使得算法提前收敛到局部最优解。另外,为了追求快速收敛,我们简单地把每一个超过容量约束的边增加1,这样粗粒度的增加,可能会加重上面讨论的拥塞循环。
\begin{figure}
\setlength{\abovecaptionskip}{-0.5cm}
\begin{center}
{\includegraphics[width=0.4 \textwidth]{figures/random.pdf}}
\end{center}
\caption{{\footnotesize{链路更新例子(2)}}}
\label{u2}
\end{figure}

如图\ref{u2}所示,假设存在两个业务$d_1$和$d_2$,其中$d_1$的源节点为$A$,目的节点为$G$,其流量大小为$5$,$d_2$ 的源节点为$B$,目的节点为$G$,其流量大小也为$5$。开始时两个都分别贪心地计算最短路径,$d_1$选择路径$A-C-E-G$,$d_2$选择路径$B-C-E-G$, 这样的话,链路$C-E$和$E-G$出现拥塞,表 \ref{Iterprocess2} 展示了这个迭代过程。图 \ref{u2}中最优的选择是让其中一个业务经过边$C-E-G$进行中继,另一个业务经过边$D-F-G$进行中继。但是迭代过程始终无法使得两条链路发生分离,图中的链路权重始终无法达到最优条件,这是因为链路$C-E-G$和链路$D-F-G$每次超限都会为路径的总权重增加2个单位,权重增加的粒度太大,假设每次迭代时链路$C-E-G$总共只增加一个单位,那么算法只需要一次迭代就能达到最优条件,此时链路$C-E-G$由于超限,权重一共增加1个单位,那么路径$A-C-E-G$ 和路径$A-D-F-G$ 权重相等都为4,同样,路径$B-C-E-G$和路径$B-D-F-G$的权重也相等了,这样两个业务才会分离开(比如业务$d_1$选择链路$A-C-E-G$,业务$d_2$选择链路$B-D-F-G$)。为了解决链路增加粒度过大的情况,在本设计中我们采用随机更新的策略,也就是对一条流量超过容量约束的边$(i,j)$,我们以概率$\phi$ 来对他进行权重更新,每条边的权重增加粒度依然为1个单位,假设$\phi=0.5$,这种方法在一定概率上保证图 \ref{u2} 中的例子可以在一次迭代中收敛。我们的实验发现这种更新策略能够在保证算法收敛速度的同时,求解到更加优化的解。
\begin{table}[t]
\newcommand{\tabincell}[2]{\begin{tabular}{@{}#1@{}}#2\end{tabular}}
\centering
\scriptsize{
\renewcommand{\tabcolsep}{0.09cm}
\renewcommand{\arraystretch}{1.2}
\caption{拉格朗日更新过程(2)}
\begin{tabular}{| c | c | c |}
\hline
Iteration number & \tabincell{l}{Calculated paths for \\the two demands} & Path weights
\\ \hline
0 & \tabincell{l}{A-C-E-G \\ B-C-E-G} & \tabincell{l}{3(1+1+1) \\ 3(1+1+1)} \\ \hline
1 & \tabincell{l}{A-D-F-G \\ B-D-F-G } & \tabincell{l}{4(2+1+1) \\ 4(2+1+1)} \\ \hline
2 & \tabincell{l}{A-C-E-G \\ B-C-E-G} & \tabincell{l}{5(1+2+2) \\ 5(1+2+2)} \\ \hline
3 & \tabincell{l}{A-D-F-G\\ B-D-F-G} & \tabincell{l}{6(2+2+2) \\ 6(2+2+2)} \\ \hline
4 & \tabincell{l}{A-C-E-G \\ B-C-E-G} & \tabincell{l}{7(1+3+3) \\ 7(1+3+3)} \\ \hline
5 & \tabincell{l}{ ...\\ ...} & \tabincell{l}{... \\ ...}\\ \hline
\end{tabular}
\vskip 0.2 cm
\label{Iterprocess2}
}
\end{table}
\subsection{路径调整}
注意到,在最优的权重代价(最优拉格朗日乘子)下,LR-PTEA求解到的路径集合解是拉格朗日对偶问题的优化解,但是它不一定是原来业务量工程问题的可行解。作为一个例子,考察图 \ref{ppt}中的情况,图中的元组中数字分别表示链路的代价和链路上的剩余容量大小,我们假设有两个业务需求$d_1$和$d_2$,其中两个业务的流量需求都是1个单位,假设在某一个迭代过程中,LR-PTEA为两个业务算出来的链路都是$A-C-E-D$, 这种路径选择是对偶问题的最优解,但是他不是原问题的可行解,因为链路$(C,E)$ 上承载的链路容量大于了链路$(C,E)$ 上的容量。这是由于每个业务在选择路径的时候没有去考虑其他业务所选择的路径,但是我们容易看到这个例子中所有路径权重都一样(都等于3),也就是说,图中存在着大量的等价链路,但是在这个例子中两个业务恰好都选择了冲突的那一条路径$A-C-E-D$,如果业务$d_1$的路径被调整到$A-D$,业务$d_2$的链路被调整到$B-F$,那么,我们可以得到原问题的最优可行解。所以通过这里的启发,我们发现需要设计一种业务路径调整算法来主动避免链路出现冲突,以得到原问题的可行解。
\begin{figure}
\setlength{\abovecaptionskip}{-0.5cm}
\begin{center}
{\includegraphics[width=0.4 \textwidth]{figures/PathAdj.pdf}}
\end{center}
\caption{{\footnotesize{路径调整例子}}}
\label{ppt}
\end{figure}

为了说明业务路径调整算法,我们引入一些符号,假设$P$表示步骤2所计算出来的路径集合,其中$p_d\in P$ 表示业务$d$ 的路径; $rp_{d}$ 表示路径$p_d$上的可用带宽,$rp_{d} = min\{r_e | e\in p_d\}$,其中$r_e$ 表示链路$e$上的剩余带宽;$D_l$ 表示剩余业务集合,表示这些业务不能在不违背容量约束的情况下被加入网络中。

路径调整算法如算法\ref{PathAdj}所示。路径调整算法的主要思想是通过调整一小部分业务的路径来得到原问题的优化可行解,算法首先对业务进行排序,这里采用3.3.1.2类似的思想对业务进行排序,一方面,要使得目标函数变小,那些流量需求较大的业务应该优先被加入到网络中,但是如果大流量的业务的路由代价很大,经过了一条很长的路径,就会大量的浪费网络中的链路容量资源,所以算法对当前解中的业务和其路径按照$\frac{bw_d}{\sqrt{|p_d|}}$的值进行排序,其中${bw_d}$代表业务$d$所需要的流量大小,$|p_d|$代表业务$d$的路径$p_d$的代价大小,这样按照顺序试图将业务加入到网络中,如果$rp_d \ge bw_d$,表示业务可以被加入到网络中,那么加入此业务并且更新网络链路的剩余容量值,反之,如果$rp_d<bw_d$,则表示业务选择的路径上的链路容量不足以承载此业务,那么将业务加入剩余集合$D_l$ 中。第2行到第9行的循环结束后,我们得到一个剩余网络$G^{'}(V^{'},E^{'})$,根据前面的讨论,在剩余网络中可能存在一些等价路径,剩余链路中依然有很多可用资源,所以算法重新在剩余网络中为剩余业务$D_l$计算路径,算法依次遍历集合$D_l$,看能否在剩余网络中为业务寻找一条路径,首先剔除那些链路剩余容量小于业务流量$bw_d$ 的链路,这样保证求出来的路径肯定是满足容量约束的。由于剩余链路是残余网络,所以可能会求出跳数很长的路径,如果跳数太长了,会占用太多的资源,不能达到优化的目的,所以我们对跳数进行约束,如果$|p|/|p_d|<\delta$,则将业务加入到网络中,并更新网络链路容量,否则不加人业务到网络中。虽然这个过程对路径的计算是串行的,但是这个过程是很快速的,这主要有以下原因:第一,由于剩余网络中的链路容量普遍较小,能参与计算的链路很少,对一个剩余业务$d$,在计算路径之前,算法会剔除那些剩余容量小于$bw_d$的链路,所以实际上参与计算的网络拓扑很小。第二,剩余业务量集合$D_l$本身较小。第三,算法越往后面加入业务,可用链路会越来越小,网络会进一步变小。通过这个路径调整算法,可以快速得到原问题的一个优化可行解。
\begin{algorithm}[t]
\begin{algorithmic}[1]
\Require
网络拓扑 $G(V, E)$;
业务量需求集合 $D$;
步骤2算出来的路径集合 $P_{in}$;
\Ensure
调整后的路径集合 $AP$;
\State {把业务按照值$\frac{bw_d}{\sqrt{|p_d|}}$进行降序排序}
\For{ $d \in D$}
\If{$rp_{d} \ge bw_d$ }
\State{把路径 $p_d$加入到$AP$中}
\State{在图$G(V,E)$中更新路径$p_d$所经过链路的剩余带宽}
\Else
\State {把业务$d$加入到剩余集合$D_l$中}
\EndIf
\EndFor
\State{$G^{'}(V^{'},E^{'})=G(V,E)$}
\For{ $d \in D_l$}
\State{$G^{''}(V^{''},E^{''})$$=$$G^{'}(V^{'},E^{'})$}
\For{ $e \in E^{'}$}
\If{链路$e的剩余容量< bw_d$}
\State{把链路 $e$ 从图$G^{''}(V^{''},E^{''})$中移除}
\EndIf
\EndFor
\State{在图$G^{''}(V^{''},E^{''})$中为业务$d$计算最短路径$p$}
\If{$\frac{|p|}{|p_d|}\le \delta$}
\State{把路径 $p$ 加入到 $AP$,并把业务从集合$D_l$中移除}
\State{在图$G^{'}(V^{'},E^{'})$中更新路径$p$所经过链路的剩余带宽}
\State{$G(V, E)=G^{'}(V^{'},E^{'})$}
\EndIf
\EndFor
\end{algorithmic}
\caption{路径调整算法}
\label{PathAdj}
\end{algorithm}

\subsection{终止条件}
LR-PTEA的终止条件和GA-PTEA的终止条件一样,假设前$k$次迭代找到的最优解为$B^*$,第$k+1$次迭代找到最优解为$b_{k+1}^*$,如果$b_{k+1}^*<B^*$,那么就更新$B^*=b_{k+1}^*$,如果连续迭代$L$次,$B^*$都不被更新,则判定算法收敛,算法停止。
\section{仿真实验分析}

\subsection {仿真介绍}

为了证明LR-PTEA和GA-PTEA的优化效果,我们把两个算法的优化结果和基于备选路径的MILP模型的最优目标值进行比较,其中每个业务的备选路径数量$K=30$条,我们使用CPLEX \citing{CPLEX}来求解MILP模型,由于CPLEX求解MILP需要很大的计算量,对于大规模的网络,我们无法得到MILP的最优值,因此我们设置了10分钟的求解时间限制,当Cplex求解时间超过10分钟后,我们停止求解,记录Cplex求得的最优可行解和Cplex估计出来的最优下界。

为了观察LR-PTEA和GA-PTEA的加速效果,我们分别设计了两种算法的串行版本LR-STEA和GA-STEA,其中LR-STEA中的路由算法采用带堆优化的dijkstra算法。

我们分别采用ERdos-Renyi (ER) \citing{ER}和 Barab asi-Albert (BA) \citing{BA}两种模型来生成实验网络拓扑,实验中的网络拓扑中点的平均度数为6,网络中的链路代价都是被设置为1。实验中我们随机产生一定数量的业务到达网络,每个业务的的流量大小被设置为1到100之间的随机值。

本文分别比较了各个算法的目标函数,加速效果和算法的收敛性质。GA-PTEA和LR-PTEA中的并行算法都是使用CUDA8.0 进行实现的,跑算法的服务器配置有四个Intel Xeon E5-2630 CPU 和一个 NVIDIA Tesla K40M GPU。

实验中的所涉及到详细参数设置如表\ref{paramter}所示。

\begin{table}[t]
\newcommand{\tabincell}[2]{\begin{tabular}{@{}#1@{}}#2\end{tabular}}
%\setlength{\abovecaptionskip}{-0.5cm}
\centering
\scriptsize{
\renewcommand{\tabcolsep}{0.09cm}
\renewcommand{\arraystretch}{1.2}
\caption{实验参数设置表}
\begin{tabular}{| p{1.2cm} | p{1.2cm} | p{1.2cm} | p{1.2cm} | p{1.2cm} | p{1.2cm} | p{1.2cm} | p{1.2cm} | p{1.2cm} | p{1.2cm} |}
\hline
W&$\delta$&$\phi$&POP&$\alpha$&$\beta$&$\gamma$&m&K&L\\ \hline
50&2&0.15&5000&500&2000&2500&1&30&30\\ \hline
\end{tabular}
\vskip 0.2 cm
\label{paramter}
}
\end{table}

\subsection{目标函数比较}
图 \ref{OB-TA}显示了在点数为1000的网络中,目标函数随着业务数量规模变化的折线图,图中的网络链路容量大小为100。从图 \ref{OB-TA}中我们可以看到目标函数大致随着业务数量呈现线性增加,这和目标函数的表达式相吻合。

从图 \ref{OB-TA}中可以看到LR-STEA/LR-PTEA的优化结果明显优于遗传算法GA-STEA/GA-PTEA的优化结果,这是因为遗传算法提前陷入了局部最优解,导致算法提前结束。

和备选路径MILP模型的Cplex解相比,LR-STEA/LR-PTEA的解明显优于Cplex的解。随着业务数量的增加,LR-STEA/LR-PTEA解和Cplex解的距离逐渐拉开,这是因为Cplex的计算压力随着业务规模的增加而增大,使得Cplex不能在有限时间内找到更好的解。

图\ref{OB-CA}表示的是在容量变化下的目标函数值的变化,其中业务数量为6000,网络链路容量从100到250变化。我们看到,随着容量的增加目标函数大致呈现线性下降。当链路容量等于250时,LR-PTEA/LR-STEA已经很接近MILP的下界了,这说明LR-PTEA/LR-STEA在网络资源足够时,能很好的优化路由代价。
\vspace{0.8cm}
\begin{figure}
\setlength{\abovecaptionskip}{-0.5cm}
\begin{center}
{\includegraphics[width=0.8 \textwidth]{figures/OB-TA.pdf}}
\end{center}
\caption{{\footnotesize{目标函数值随业务数量变化示意图}}}
\label{OB-TA}
\end{figure}
\begin{figure}
%\vspace{-0.2cm}
\setlength{\abovecaptionskip}{-0.5cm}
\begin{center}
{\includegraphics[width=0.8 \textwidth]{figures/OB-CA.pdf}}
\end{center}
\caption{{\footnotesize{目标函数值随链路容量变化示意图}}}
\label{OB-CA}
\end{figure}

\begin{figure}
\setlength{\abovecaptionskip}{-0.5cm}
\begin{center}
{\includegraphics[width=0.8\textwidth]{figures/TI-ER-TA-1000.pdf}}
\end{center}
\caption{{\footnotesize{计算时间随业务数量变化示意图(ER)}}}
\label{TI-ER-TA-1000}
\end{figure}
\begin{figure} 
\setlength{\abovecaptionskip}{-0.5cm}
\begin{center}
{\includegraphics[width=0.8\textwidth]{figures/TI-BA-TA-1000.pdf}}
\end{center}
\caption{{\footnotesize{计算时间随业务数量变化示意图(ER)}}}
\label{TI-BA-TA-1000}
\end{figure}

\subsection{算法时间比较}
图 \ref{TI-ER-TA-1000}和图 \ref{TI-BA-TA-1000}分别列出了各个算法在BA和ER拓扑下的计算时间随着业务数量增加的变化情况,图中拓扑节点数量为1000,业务数量从1000 到6000变化,链路的容量为100。由于各种算法的时间差距过大,我们用三种不同的尺度进行展示。首先,Cplex的时间限制都为10分钟,所以我们不再列出来。

在图\ref{TI-ER-TA-1000}和图 \ref{TI-BA-TA-1000}中我们可以看到GA-PTEA相对于GA-STEA加速比可达到20倍以上,但是由于遗传算法本身收敛较慢,所以GA-PTEA的计算时间还是很糟糕的。

观察图 \ref{TI-ER-TA-1000}和图 \ref{TI-BA-TA-1000},我们发现LR-STEA和LR-PTEA的时间曲线会发生大幅度的波动,这是因为算法的迭代次数变化较大。根据3.4.5的讨论,在连续$L$次没有寻找到更优解后,LR-PTEA/LR-STEA就会停止,我们把$L$设置为30,其实$L$设置得偏小了,算法有可能会找到更好的解,但是这些解的下降程度很小(通过收敛图 \ref{LR-CO},可以看到算法在前30次内下降幅度较大,后面的下降幅度很小),所以为了优化整体时间,我们设置迭代次数为较小的30次,这样可能会使得算法提前结束,从而导致算法的迭代次数变化幅度较大。

观察图 \ref{TI-ER-TA-1000}和图 \ref{TI-BA-TA-1000},我们发现随着业务数量的增加,LR-PTEA的加速优势变大。总体来看,LR-PTEA相对与LR-STEA有平均5-6倍的加速。


图 \ref{TI-BA-NO}和图 \ref{TI-ER-NO}表示算法计算时间随着网络拓扑大小的变化情况,其中到达网络的业务数量为网络拓扑点数目的4倍,可以看到随着网络拓扑变大,计算时间也相应增加,GA-PTEA的计算时间不管在小拓扑还是大拓扑下都远远大于LR-PTEA 的计算时间。拓扑较小时LR-PTEA相对与LR-STEA的加速很小,但随着网络拓扑规模的增加,LR-STEA的计算时间大幅上升,而采用GPU加速计算的LR-PTEA上升幅度较小,网络拓扑达到一定规模后LR-PTEA对LR-STEA有较大的加速优势。

\begin{figure}
\setlength{\abovecaptionskip}{-0.5cm}
\begin{center}
{\includegraphics[width=0.8\textwidth]{figures/TI-BA-NO.pdf}}
\end{center}
\caption{{\footnotesize{计算时间随拓扑大小变化示意图(BA)}}}
\label{TI-BA-NO}
\end{figure}
\subsection{算法收敛性}
图 \ref{GA-CO} 和图 \ref{LR-CO}分别是GA-PTEA和LR-PTEA的收敛过程图。可以看到遗传算法的初始值较差,迭代次数较多,而且最终陷入到了局部最优解,而LR-PTEA的算法第一次的解已经很好,这是因为LR-PTEA不是基于备选路径的,他会通过路径调整策略为业务重新寻找路径,从而得到一个较好的初始目标值,LR-PTEA的目标函数值在很短的迭代次数里快速下降,但是LR-PTEA的曲线不够平滑,说明算法的波动较大,收敛终止条件不好控制,这也是图 \ref{TI-ER-TA-1000}和图\ref{TI-BA-TA-1000}产生波动的原因。

\section{本章总结}
本章研究了SDN IP网络下的并行业务量工程算法,首先提出了新的业务量工程效用函数,建立了业务量工程模型,设计了两种基于GPU的并行优化算法方案GA-PTEA和LR-PTEA,我们的实验发现GA-PTEA的加速比可达20倍,LR-PTEA的加速比能达到6倍,LR-PTEA能够在短时间内得到目标函数的优化解。

\begin{figure}
\vspace{-0.2cm} 
\setlength{\abovecaptionskip}{-0.5cm}
\begin{center}
{\includegraphics[width=0.8\textwidth]{figures/TI-ER-NO.pdf}}
\end{center}
\caption{{\footnotesize{计算时间随拓扑大小变化示意图(ER)}}}
\label{TI-ER-NO}
\end{figure}
\begin{figure}
\vspace{-0.2cm} 
\setlength{\abovecaptionskip}{-0.5cm}
\begin{center}
{\includegraphics[width=0.8\textwidth]{figures/GA-CO.pdf}}
\end{center}
\caption{{\footnotesize{GA-PTEA 收敛性}}}
\label{GA-CO}
\end{figure}
\begin{figure}
\setlength{\abovecaptionskip}{-0.5cm}
\begin{center}
{\includegraphics[width=0.8\textwidth]{figures/LR-CO.pdf}}
\end{center}
\caption{{\footnotesize{LR-PTEA 收敛性}}}
\label{LR-CO}
\end{figure}






